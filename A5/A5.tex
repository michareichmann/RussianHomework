\documentclass[a4paper]{article}
% \usepackage[utf8]{inputenc}
\usepackage{fontspec}
\usepackage{polyglossia}
\newfontfamily{\cyrillicfont}{Carlito}  %{Liberation Mono}
\setmainlanguage{russian} 
\setotherlanguage{english}
% \usepackage[russian]{babel}
\usepackage{fancyhdr}% http://ctan.org/pkg/fancyhdr
\pagestyle{fancy}% Change page style to fancy
\fancyhf{}% Clear header/footer
\fancyhead[C]{\textit{Portfolioaufgabe 5}\hfill Michael Reichmann}
\usepackage[printonlyused]{acronym}
\usepackage{siunitx}
\usepackage{graphicx}
\usepackage{pgfpages}
\usepackage{subcaption}
\usepackage{varioref}
\usepackage{url}
\newcommand{\as}{\\[14pt]}
\newcommand{\s}{\\[7pt]}
\newcommand{\is}{\\[2pt]}
\newcommand{\no}{\noindent}
\newcommand{\ka}{\hspace*{0.5cm}}
\newcommand{\ma}{\hspace*{1cm}}
\newcommand{\ga}{\hspace*{1.5cm}}
\newcommand{\li}{\left|}
\newcommand{\re}{\right|}
\newcommand{\const}{\text{const.}}
\newcommand{\z}{\text}
\newcommand{\terminal}[1]{\colorbox{black}{\textcolor{white}{{\fontfamily{phv}\selectfont \scriptsize{#1}}}}}
\newcommand{\plugin}[1]{\textit{\flq#1\frq}}
\newcommand{\ra}{$\rightarrow$ }
\definecolor{cadmiumgreen}{rgb}{0.0, 0.42, 0.24}
\newcommand{\itemfill}{\setlength{\itemsep}{\fill}}
\newcommand{\orderof}[1]{$\mathcal{O}\left(#1\right)$}
\newcommand{\fig}[2]{\begin{figure}\centering\includegraphics[height={#2}\textheight]{#1}\end{figure}}
\newcommand{\figc}[3]{\begin{figure}\centering\includegraphics[height={#2}\textheight]{#1}\caption{#3}\end{figure}}
\newcommand{\figp}[2]{\begin{figure}\centering\includegraphics[width={#2}\textheight, angle=-90]{#1}\end{figure}}
\newcommand{\figpc}[3]{\begin{figure}\centering\includegraphics[width={#2}\textheight, angle=-90]{#1}\caption{#3}\end{figure}}
\newcommand{\test}[1][bla]{#1}
\newcommand{\subfig}[4][0.45]{\begin{subfigure}{{#1}\textwidth}\centering
			\includegraphics[height={#3}\textheight]{#2}
			\caption{#4}\end{subfigure}}
\newcommand{\subfigp}[4][0.45]{\begin{subfigure}{{#1}\textwidth}\centering
			\includegraphics[width={#3}\textheight, angle=-90]{#2}
			\caption{#4}\end{subfigure}}

\usepackage{amsthm}
\theoremstyle{plain}
%\newtheorem*{def}

\makeindex
\begin{document}

\begin{description}
 	\item[\textcolor{red}{Aufgabe 5:}] Schreiben Sie ein kleines Gespräch bei einem gemeinsamen Essen\linebreak verschiedener Personen anlässlich eines Geburtstags (ca. 80 Wörter)!\linebreak Verwenden Sie dabei das sprachliche Material aus der Lektion 1\linebreak (Gratulationen, Trinksprüche, usw.)!

\end{description}
\vspace*{1cm}


\begin{itemize}
	\item[все поют:] 	С днём рождения тебя,\\
						С днём рождения тебя,\\
						С днём рождения, милый михаэль,\\
						С днём рождения тебя!
	\item Спасибо большое!
	\item[$\circ$] за твоё здоровье!
	\item Этот пиво очень хорошо!
	\item[$\diamond$] Я желаю тебе всего наилучшего, успехов и здоровья!
	\item[$\triangleright$] А я желаю тебе счастья и много любви!
	\item Спасибо, спасибо, друзья!
	\item[$\circ$] Между первой и второй перерывчик небольшое! За прекрасных дам! 
	\item Вы знаете что вы хотите заказать? Я приглашаю вас на ужин!
	\item[$\triangleright$] Это при-классно! За знакомство!
	\item[$\diamond$] Я хочу заказать борщ! А вы хочешь пить рюмку водки?
	\item[$\circ$] Да! Конечно! Я тоже буду есть борщ.
	\item[$\triangleright$] А я хочу есть пельмени!
	\item Вот водка и еда! За здоровье и приятного аппетита! 
\end{itemize}

\end{document}
